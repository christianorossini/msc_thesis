\vspace{2.2cm}

\begin{center}
    \noindent{\Huge \textbf{Resumo}}
\end{center}

%\vspace{1.6cm}
\vspace{0.5cm}

\noindent O \textit{code smell} é um sintoma de um mau design de código em desenvolvimento de software. Geralmente ocorre quando os desenvolvedores conceberam mal o design de um componente de código ou porque não projetaram a solução adequadamente devido a prazos rígidos. Na literatura, o \textit{code smell} é descrito informalmente, isto é, não é possível identificá-lo objetivamente. Essa informalidade pode levar dois ou mais desenvolvedores a raciocinar sobre cada ocorrência de \textit{smell} a sua maneira. Como conseqüência de diferentes pontos de vista, percepções conflitantes nas mesmas bases de código podem ser notáveis, afetando a consistência entre as revisões de código. 
Trabalhos anteriores que abordam a subjetividade do \textit{code smell} carecem de elementos informativos que possam descrever por que certos códigos-fonte foram anteriormente classificados como \textit{code smell}, a fim de ajudar os desenvolvedores a raciocinar sobre a ocorrência de um \textit{code smell} com mais eficácia. Em nossa pesquisa, propomos mostrar ao desenvolvedor uma visualização de um classificador de árvore de decisão, composto por regras baseadas em métricas de software, com o objetivo de informar as razões pelas quais algum código-fonte foi classificado anteriormente como um \textit{host} de um \textit{code smell}. O fornecimento de novos \textit{insights} pode levar o desenvolvedor a raciocinar de forma mais ampla sobre a ocorrência de um \textit{code smell}, não se restringindo a fatores relacionados apenas a experiências e vida profissional. Nosso objetivo é, após várias avaliações de \textit{code smells}, investigar como a concordância entre desenvolvedores pode ser influenciada pela visualização fornecida por um modelo de classificação compreensível, o classificador de árvore de decisão. Realizamos um experimento on-line onde reunimos colaborações de 30 desenvolvedores da indústria e da academia. Os resultados indicam que: (i) a detecção de \textit{code smells} auxiliada por uma árvore de decisão leva a uma melhora relativa da concordância em relação às detecções com base apenas na análise de código; (ii) a detecção de \textit{code smell} auxiliada pela árvore de decisão não diminui o esforço para detectar \textit{smells} (iii) nosso experimento sugere que as árvores de decisão usadas para dar suporte à detecção de \textit{code smells} são úteis para o desenvolvedor em termos de \textit{insights} para tomada de decisão.


\textbf{Palavras-chave}: Odores de código, Concordâncias, Revisão de código, Árvore de Decisão, Estudo empírico