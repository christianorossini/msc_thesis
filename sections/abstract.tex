\vspace{2.2cm}

\begin{center}
    \noindent{\Huge \textbf{Abstract}}
\end{center}

\vspace{1.6cm}

\noindent Code smell is a symptom of poor code design on software development. It often occurs when developers poorly conceived the design of the code component or because they did not properly designed the solution due to strict deadlines. In literature, code smell is described informally, i. e., it isn't possible to identify a smelly code objectively. Such informality may lead two or more developers to reason about each smell occurrence in their own way. As a consequence of different viewpoints, conflicting perceptions on the same code bases may be notable, impacting the consistency across code reviews. Previous works that addresses code smell subjectivity lacks of some informative element that may describe why certain suspicions source code was previously classified as code smell in order to aid developers to reasoning about the occurrence of a code smell with more effectivity. In our research, we propose to show to the developer a visualization of a decision tree classifier, composed by some metric-based rules, aimed to inform the developer the reasons why some source code was previously classified as a host of a certain code smell. Providing new insights may tease the developer to reasoning the occurrence of a code smell widely, not restricting to factors concerned only to past experiences and backgrounds. Our objective is, after several code smell evaluations, investigate how the agreements among developers may be influenced by the visualization provided by a comprehensible classification model, the decision tree classifier. We performed an on line experiment where we gather collaborations from 30  developers from industry and academy. The results indicate: (i) code smells detection aided by a decision tree leads to a relative improvement of agreement in relation to detections based solely on code analysis; (ii) the detection of code smell aided by decision tree do not decrease the effort to detect smells (iii) our experiment suggests that the decision trees used to support code smell detection are useful to the developer in terms of insights for decision making.
 
 
\vspace*{0.5cm}\noindent\textbf{Keywords}: Code Smell, Agreements, Code Review, Decision Tree Classifier, Detection, Empirical study