\chapter{Related Work}
\label{sec:relateds}

To the best of our knowledge, this is the first work that utilizes a comprehensible machine learning classifier to aid developers to detect code smell and how this approach deals with the agreement among them. However, there are some studies that investigate the agreement among developers when detecting code smells and the factors that influence their evaluations. The decision tree algorithm, which is an important part of this study, was also included in the literature addressing the effectiveness of the automatic code smell detection. 

In \cite{mantyla2005experiment}, the author presented a study that investigated the agreement among developers about 3 types of code smells: Long Method, Long Parameter List and Feature Envy. The results indicated a low agreement related to the Feature Envy, but considerable agreement on Long Method and Long Parameter List evaluations. According to the authors, these smells are easier to understand and detect, making the evaluations more similar. The results showed that the number of lines of code ( MLOC ) and the number of parameters in a method (NPARAM) could be used as predictors of the evaluations related to Long Method and Long Parameter List, respectively. Even though the results of the study reported interesting findings concerning how similar the developers detect smells in code, the study analyzed the agreement among developers on evaluating only 3 smell types. 

Mika Mäntylä et al. \cite{mantyla2006subjective} extended the previous study by involving 12 developers analyzing code snippets of 23 types of code smells. The author presented a definition and an example of each type of code smells to the developers, and then he asked them to evaluate some code snippets related to the analyzed code smell. The participants of the empirical study evaluated the presence or absence of each smell across several code snippets. The results showed a perfect agreement of 5 developers in only 1 of 46 analyzed code snippets presented in their publication, concerning the analysis of a Long Method smell. Otherwise, the authors reported the highest disagreement in the evaluations of the Switch Statement, Inappropriate Intimacy , and Message Chains smells. Although the interesting findings regarding the agreement among developers, the authors stated that the data used in the experiment were not sufficient to support more confidence regarding the results. The reduced number of analyzed code snippets and developers involved in the experiment made difficult to verify the statistical significance of the agreement among the developers. 

Hozano et al. \cite{hozano2018you} presented a study aimed to investigate how similar the developers detect smells in code and identify possible factors that influence in the agreement of their evaluations. They made an empirical study involving 75 developers evaluating 15 types of code smells into code snippets of 5 open source projects. In total, more than 2700 evaluations of code snippets from real software projects were performed. From the collected evaluations, they identified a great disagreement among the developers’ evaluations considering the code smells investigated in the study. The results evidenced, in general, that developers detect code smells in different ways. In contrast, developers that followed the same heuristic to evaluate a given code smell presented a consistent agreement when analyzed separately, so the authors stated that the heuristics plays an important role to determine the agreement among the developers.

The effectiveness of the decision tree on code smell detection was addressed by Amorim et al. \cite{amorim2015experience}. In their work, they presented a practical experience report that evaluates the use of Decision Tree classification algorithm to recognize code smells in different software projects. They aimed at contributing to the state-of-the-art on code smells detection by analyzing the use of the Decision Tree algorithm to detect code smells automatically. They evaluated the performance of the Decision Tree to detect 12 types of different smells across 4 open source projects.  They compared the precision, recall and F-measure results with other machine learning techniques, such as SVM and Bayesian Beliefs Networks, as well as the results generated by general rules-based approaches. For both cases, the decision tree reaches better performance. 


