\chapter{Conclusion and Future Works}
\label{sec:conclusion}

In this research, an empirical study was presented to investigate the agreement among developers when they are aided by a comprehensible classification model to support decision on smell detection. Following the same way, the study investigates the required effort to perform evaluations by visualizing the classifier model and the usefulness of our approach by participant perspective. 

To answer the raised questions, we made an empirical study involving 30 participants who evaluate 8 tasks, divided into 2 groups, with source codes potentially affected by 1 of 4 selected types of smell. Each group of tasks exposed the developer to a source code with different perspectives, so one shows the classification model that predicted the code as "smelly" whereas another group doesn’t.  The experiment was carried out through a web-based app developed exclusively to collect the evaluations performed by participants. In total, more than 230 evaluations of code snippets from real software projects were performed.

Throughout our work, we got evidence that code smells detection aided by a decision tree leads to a relative improvement of agreement in relation to detections based solely on pure and simple code analysis, although in both scenarios the resulting Kappa measure was considered slight regarding the agreement strength. After detaching different profiles of participants from the whole set of participants, we discovered groups that behave distinctly. The agreement on the detections aided by decision tree performed better with experienced code smell detection participants, which brings the evidence that such profile got the best out of our approach. As for the effort, the detection of code smell aided by decision tree did not decrease in time compared to detection based only on code inspection. For both groups of tasks, there isn’t any evidence that indicates the benefits related to the effort reduction when detecting code smells with decision tree, i. e., the time spent to detect smells aided by decision tree tend to  be equivalent or to be highest than  the time spent to detect code smells based solely in code inspection. Finally, based on the answers provided by the majority of participants, our experiment suggests that the decision trees used to support code smell detection are useful to the developer in terms of insights to decision making.

As future work, we intend to use the feedback from participants to improve the generated models and made it more useful to detecting smells. Moreover, in the next works we intend to cover a wider range of code smells and study new agreements covering these smells.


